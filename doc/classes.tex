\section{Structure of the New Code}

This document outlines the major groups of classes in our code.

\subsection{Overall Structure}

\begin{itemize}
    \item \textbf{Controllers} implement the core logic of the application. They are responsible for maintaining the state of the view and redrawing the view on any updates. They utilize necessary helper classes to ensure their aim.
          \begin{itemize}
              \item ControlDesk
              \item Pinsetter
              \item Lane
          \end{itemize}
    \item \textbf{Views} are concerned with providing a GUI to the end user, and do not implement any logic. They receive instructions from Controller clases via Observer pattern.
          \begin{itemize}
              \item AddPartyView (No controller)
              \item AdhocView (No controller)
              \item BowlerScorerView
              \item ControlDeskView
              \item LaneStatusView
              \item LaneView
              \item NewPatronView
              \item PinsetterView
              \item EndGamePrompt
              \item EndGameReport
          \end{itemize}
    \item \textbf{Controller helpers} classes provide necessary abstracted help to the main controller classes, as the following, and both are described later in detail.
          \begin{itemize}
              \item ScorableParty
              \item ScorableBowler
          \end{itemize}
    \item The Events broadcasted as a part of the observer pattern are done through are done through the `XEvent.java' classes. It wraps all data needed by the observers from the publisher.
          \begin{itemize}
              \item LaneEvent
              \item PinsetterEvent
              \item ControlDeskEvent
          \end{itemize}
    \item \textbf{View helper} classes that help the main View classes in displaying certain specific window components, like a panel of buttons or a grid panel, etc.
          \begin{itemize}
              \item ButtonPanel
              \item ConainerPanel
              \item FormPanel
              \item GridPanel
              \item SrollablePanel
              \item TextFieldPanel
              \item Generic Panel (Abstract superclass)
              \item WindowFrame
          \end{itemize}
    \item \textbf{Abstract classes} that layout the necessary functions for classes to implement and decouple our implementations from the requirements due to it's users.
          \begin{itemize}
              \item Observer
              \item Publisher
              \item Event
              \item LaneWithPinsetter
          \end{itemize}
    \item \textbf{Utility classes} that help with Input/Output and maintaining data, often for specific classes.
          \begin{itemize}
              \item Util
              \item PrintableText
              \item BowlerFile
              \item ScoreHistoryFile
              \item ScoreReport
          \end{itemize}
    \item Finally, we have the \textbf{driver class}, which initializes an Alley with the corrrect parameters and lets it run:
          \begin{itemize}
              \item BowlingAlleyDriver
          \end{itemize}
    \item Lastly, there is a single Test file, to automate correctness checks of the lane scoring mechanism. We have not added more as it's not an everchanging piece of code, most of the testing is done through asserts, and throwing errors under unexpected events.
          \begin{itemize}
              \item BowlerScorerTest
          \end{itemize}
\end{itemize}

\subsection{Responsibilities of each major class}

\begin{enumerate}
    \item \code{Lane}: Manages an entire party and its scoring on a lane. It makes sure bowlers get to bowl one after the other, are able to play multiple games one after another, and in the end can exit the lane while knowing their scores.
    \item \code{ControlDesk}: Allows parties to be added to the alley, wait in the queue to be assigned a lane, and then assign them to a free lane.
    \item \code{ScorableParty}: a specialization of a party, maintains a list of bowlers, and also manages final cumulative scores and per game scores. Acts as a \textbf{mediator} between Lane and a collection of ScorableBowler.
    \item \code{ScorableBowler}: a specialization of a Bowler, in addition to being able to bowl balls down an alley, it can also keep track of its per game score.
    \item \code{Widget} package: maintains several (abstract) classes that \textbf{significantly simplify} the setting up of Views in our project.
    \item \code{Pinsetter}: maintains the state of a pinsetter per lane, and ensures a valid number of pins fall down every time a bowler rolls.
\end{enumerate}


\subsection{Control Flow}

The Control Start from \textbf{Alley} which containes the main function and starts running the program. This runs \textbf{ControlDesk} (assigning a thread to it) and \textbf{ControlDeskView}, while also subscribing ControlDeskView to ControlDesk. This communication will occur via \textbf{ControlDeskEvent}.

\subsubsection{The Control Desk}

The ControlDeskView has several buttons, which shall now describe the flow of Control. \textbf{AddPartyView} is the most important of them. It creates a \textbf{Party} of \textbf{Bowlers}, which the ControlDesk assigns to \textbf{Lane} on a Separate Thread. Lane adds more information about scoring to them by converting Party into \textbf{ScorableParty} which is a collection of \textbf{ScorableBowler} (from Bowlers). In addition, there is a \textbf{NewPatronView} that allows the creation of new players via a text form. Each lane that gets a ScorableParty assigned to it now has a separate thread that it runs on. \textbf{AdhocView} for Queries is also available from here.

\subsubsection{The Lane}

Lane inherits from \textbf{LaneWithPinsetter}, which is an empty lane without any players or score, but it can still support the existance of bowling pins and a single throw using the \textbf{Pinsetter} class. The Pinsetter is responsible for generating these throws for any player, and keeping this process going. It reports back with the number of pins down on each throw in a \textbf{PinsetterEvent}. The Reporting to both Lane and \textbf{PinsetterView} (which is a UI element to see the state of the pins available from lane view).

\subsubsection{The Scoring}

Now coming inside Lane, Lane also has a ScorerableParty, and it's job is to relay events from the Pinsetter to the Party, and handle other jobs coming from the ControlDesk, and UI elements like Load/Save, etc. The central function in \textbf{ScorableBowler} (ScorableParty is a collection of ScorableBowler) is `roll()', and it keeps an array called `rolls' to maintain state, as well as other useful score caches (eg. total score of frame). All of this is relayed to \textbf{BowlerScorerView}, which is a part of \textbf{LaneView} (LaneView has a BowlerScorerView for each bowler). All events from Pinsetter are also relayed to \textbf{LaneStatusView} (instantiated by ControlDesk with Lane), which keeps tabs on the pins down in the current frame and can open the LaneView and PinsetterView. The Scoring logic in BowlerScorer also takes help of 2 more classes, \textbf{Frame}, and it's child \textbf{LastFrame}.

\subsubsection{Others}

Everything else is auxillary. \textbf{EndGamePrompt} and \textbf{EndGameReport} are both view classes that ask the user to continue and show him the results. There are two FileHelper classes, a Utility Class.
Finally there are some classes for the Observer patter to aid communication and the Widgets library to make working with UI smooth.

\subsection{Describing Specific Classes}

\subsubsection{AddPartyView}
GUI for adding a new party to the control desk. This supports being able to add or remove patrons from a party, or enlist a new patron and then add him.

\subsubsection{Alley}
Represents an Alley, it's a stub class given it basically integrates ControlDesk and its View together, but given our requirements, this is fine.

\subsubsection{BowlerFile}
Manages the backend of persistently storing patron nickk names, emails, and full names. Has public methods for adding a new patron, searching for existing patron, etc.

\subsubsection{Bowler}
Bowler which can roll balls and keep track of its own frame.

\subsubsection{BowlerFile}
Bowler that keeps track of all the patrons in a file.

\subsubsection{BowlerFile}
Stub class that tracks a bowler's metadata - name, email, and fullname.

\subsubsection{BowlerScorerTest}
Testing file for the scoring mechanism in bowling. Does different types of rolls and matches the expected cumulativbe score.

\subsubsection{BowlerScoreView}
GUI that renders a horizontal table of around 20 cells that update everytime that bowler makes a throw. It is unique for each bowler.

\subsubsection{ButtonNames}
Utility class for all the different buttons names on all different modals.

\subsubsection{ControlDesk}
Manages the setting up of all the lanes and the buttons and the party queue.

\subsubsection{ControlDeskEvent}
Event that is supplied to control desk's observers

\subsubsection{ControlDeskView}
The view for the control desk (figure 1 in section 2)

\subsubsection{EndGamePrompt}
The prompt that asks a party whether to replay a game or whether to exit the alley.

\subsubsection{EndGameReport}
A modal asking user to print report or not, and then exit the alley.

\subsubsection{Event}
A marker interface for the different events.

\subsubsection{Frame}
A single frame in the bowler's scoring. Each frame handles its own rolls and score contribution.

\subsubsection{Lane}
Major class managing the party in a lane, cycling through its bowlers, sending events to views, asking for prompts on game end.

\subsubsection{LaneEvent}
Helpful information for observers of lane contained here.

\subsubsection{LaneStatusView}
Displayed in control desk in figure 1 for each lane.

\subsubsection{LaneView}
The dynamic tabular view of a lane in a new window.

\subsubsection{LaneWithPinsetter}
A simple lane class that manages all the interactions of a lane with the pinsetter. This is a non-scorable lane, it does not handle scoring, which is handled by lane.

\subsubsection{LastFrame}
A specialization of the Frame class to handle certain special scoring conditions in the last frame since it has three possible rolls.

\subsubsection{NewPatronView}
Figure 3b in the pdf, adding new patron to a party.

\subsubsection{Observer}
General interface that helps us implement an observer pattern in our code.

\subsubsection{Party}
A stub class that keeps information of its bowlers.

\subsubsection{PinsetterEvent}
Event published by a pinsetter everytime pins fall in it.

\subsubsection{PinsetterView}
The view for the pinsetter where it displays fallen pins.

\subsubsection{PrintableText}
Utility class for printing text to screen.

\subsubsection{Publisher}
Abstract class for implementing the observer pattern, keeps track of subscribers of a class.

\subsubsection{ScorableBowler}
A bowler specialization that can also keep track of its scoring, specifically its final scores after every game end

\subsubsection{ScorableParty}
A party specialization with added scope for scoring wherein all the bowlers keep track of their final scores.

\subsubsection{Score}
Stub class to keep information about the score on a particular date by a particular bowler.

\subsubsection{ScoreHistoryFile}
Utility class for writing scores to a dedicated file, retrieving scores from that file, and answering ad-hoc queries.

\subsubsection{ScoreReport}
Utility class to write scores and send them in an email

\subsubsection{Util}
Other common utility methods used consistently by other classes.